\documentclass[12pt]{article}

%\usepackage[pdftex,active,tightpage]{preview}

\usepackage{skak}
\pagestyle{empty}
\parindent=0cm
\addtolength{\parskip}{0.1 in}

\begin{document}

\begin{section}*{Milton Keynes A\,--\,Milton Keynes B \\ A Stunning Start}\end{section}
Milton Keynes~B marked its return to first division with a clash against our A~team. As they were the runners up in the league last year, it looked like we were in for a tough night.\\[2mm]

\paragraph*{Board 1}Andrew Bowler\,--\,Dominic Bartram\\
A game with Dominic always promises something interesting, and tonight was no exception when he played the very rare Balogh's Defence: {\bf 1.\wmove{e4} \wmove{d6} 2.\wmove{d4} \wmove{f5!?}}. Whilst not totally unfamiliar with this defence, having experimented with it in offhand games as a teenager, I was on my own after the moves {\bf 3.\wmove{Nc3} \wmove{Nf6} 4.\wmove{Bd3} \wmove{Nc6!?}}, as I thought my fourth move forced Black to exchange in the centre. Not wanting to give up my lovely centre, I now wasted some time before playing {\bf 5.\wmove{exf5}}. This resulted in a sharp position, but I think White has an advantage. Neither of us played the position perfectly, but we eventually reached a position that I thought was winning for me when Dominic came up with the idea of giving up a piece to get a lot of activity for his queen and knight. With my king stuck in the centre, I was beginning to get a little concerned, but then Dominic overlooked a bishop sacrifice which required him to give up the queen to avoid mate, and so he resigned.


\paragraph*{Board 2} Adrian Elwin\,--\,Alan Heath\\
It was no surprise to see a \wmove{Bg5} QGD appearing on the board, and when Alan played \bmove{Nh5} to try and simplify things, Adrian was able to trade down to a position where his rooks and knight were all more active than Alan's rooks and bishop. At this stage I expected Adrian to win, especially once he had simplified things further to a rook and pawn ending where he was a pawn to the good. I don't know what happened next, but Adrian must have done something wrong as, the next time I looked at the position, Alan was two pawns up. Worse still, these two extra pawns were passed and running down the board. With no option but to play on, Adrian took Alan's b-pawn giving him two passed pawns on the queen side. Unfortunately for Adrian, whilst he was doing this Alan had advanced both of his pawns close to promotion so that Adrian had to give up his rook as one of them queened. In the resulting position, Adrian's passed pawns were too far back to cause any trouble for Alan's king and rook, an so Adrian conceded defeat.

\paragraph*{Board 3} Ozan Senturk\,--\,Eric Meichel\\
I am not certain how this game started, but it looked like a fianchetto variation of a King's Indian Defence. I don't have much of a grasp of this opening, but Ozan seemed to have a space advantage, and Eric wasn't able to play the usual f5 break. Still it was a very closed position, so it seemed like it would require a lot of work for White to make significant progress. However, the next time I checked Ozan was a piece up, and even though the position remained closed, Ozan eventually managed to break through and win the game. An impressive debut for the new player in our team. Welcome Ozan!

\paragraph*{Board 4} Graham Smith\,--\,George Ward\\
George played a Caro-Kann Defence, but after {\bf 1.\wmove{e4} \wmove{c6} 2.\wmove{d3}} George played {\bf 2\bmove{g6}}, so it ended up looking more like a Modern Defence. The game continued with both sides hardly engaging with each other. The players managed to swap queens and a couple of minor pieces, but the position remained quite closed in nature, and a draw was soon agreed. A fair result, but I think Graham had a slight plus in the final position, so maybe he could have played on a bit longer at no risk.

\paragraph*{Board 5} Sumit Bhalla\,--\,Rob Whiteside\\
This game started with a London System against the King's Indian, or what I  believe is called the Barry Attack. Whatever the name, it didn't work out for Sumit as he lost a key central pawn very early on, and when I next looked Rob played a small tactic to nab another pawn. With two extra central pawns dominating the position, it seemed only a matter of time before Rob would notch up the full point. The last time I looked at the position, Rob appeared to have a promising rook sacrifice on b2 to open up Sumit's king. I don't know whether he played this, but Rob's queen side attack soon broke through and Sumit had little choice but to resign.

So, at the end of the night, we ended up winning the match \(3\frac12\)--\(1\frac12\), for a fantastic start to the new season.




\end{document} 